\section{Computational Modeling Techniques}

Existing computational models for organoids are predominantly grouped into two categories, continuum models and cell-based models (CBMs). Continuum approaches treat the tissue as a continuous material, using differential equations to describe large-scale events like molecular diffusion or fluid flow. Conversely, CBMs capture individual cell-level mechanics. Common CMBs such as Vertex Models and Node-Based Models represent cells as distinct entities with defined interactions, allowing for numerical simulation of forces, adhesion, and morphogenesis \cite{Tanida2025}. 

CBMs often use simple ECM models, representing its mechanical influence as an isotropic stiffness parameter or external pressure resisting organoid expansion \cite{Fletcher2014}. However, more sophisticated models incorporate the ECM explicitly, with a continuous domain surrounding the cells \cite{Zeng2006}. This domain is defined by constitutive mechanical laws (e.g., linear elasticity or viscoelasticity), enabling the model to calculate reciprocal Cell-ECM interactions. For example, cell-generated traction forces deform the matrix, and the matrix resistance simultaneously constrains the organoid's morphology.

During the development of the computational model in this project thus far, the following key constitutive equations and dynamics have been used.

\subsection{Off-Lattice Node-Based Model}

% Based on: Meineke et al. 2001, Drasdo et al. 2005
% CHASTE classes: GeneralisedLinearSpringForce, OffLatticeSimulation

\paragraph{Cell-Cell Spring Forces}
The total force acting on cell $i$ from all neighbouring cells $j$:

\begin{equation}
    \mathbf{F}_i = \sum_{j \in \text{neighbors}(i)} \mu (\left| \mathbf{r}_{i,j} \right| - l_{i,j}^0) \mathbf{\hat{r}}_{i,j}(t)
    \label{eq:spring_force}
\end{equation}

\begin{itemize}
    \item $\mu$ = spring constant (stiffness)
    \item $\mathbf{r}_{i,j}$ = vector from cell $i$ to cell $j$
    \item $\mathbf{\hat{r}}_{i,j}(t)$ = unit vector in direction from $i$ to $j$
    \item $l_{i,j}^0$ = equilibrium spring length (rest length)
    \item \textbf{Origin:} Meineke et al., 2001 \cite{Meineke2001}
    \item \textbf{Bullet points to fill:}
    \begin{itemize}
        \item []
        \item []
        \item []
    \end{itemize}
\end{itemize}

\paragraph{Overdamped Motion}
Friction-dominated displacement in small time interval $\Delta t$:

\begin{equation}
    \Delta \mathbf{r}_i = \frac{\Delta t}{\eta} \mathbf{F}_i
    \label{eq:overdamped}
\end{equation}

\begin{itemize}
    \item $\mathbf{r}_i$ = position vector of cell center $i$
    \item $\eta$ = damping constant (drag coefficient)
    \item Assumes low Reynolds number, inertia negligible
    \item \textbf{Origin:} Drasdo et al., 2005 \cite{Drasdo2005}
    \item \textbf{Bullet points to fill:}
    \begin{itemize}
        \item []
        \item []
        \item []
    \end{itemize}
\end{itemize}

\paragraph{Neighbour Detection}
\begin{itemize}
    \item After each displacement step, Delaunay triangulation is re-evaluated to determine cell neighbors
    \item Cutoff distance typically set to maximum interaction range (e.g., 50 μm)
    \item \textbf{Bullet points to fill:}
    \begin{itemize}
        \item []
        \item []
    \end{itemize}
\end{itemize}

\subsection{Differential Adhesion Force}

% Based on: Differential Adhesion Hypothesis (Steinberg, 1963), implemented as extension of GeneralisedLinearSpringForce

\paragraph{Type-Dependent Spring Constant}
The spring constant $\mu$ in Equation \ref{eq:spring_force} varies based on cell types:

\begin{equation}
    \mu_{ij} = \begin{cases}
        \mu_{\text{apical-apical}} & \text{if cells } i, j \text{ both apical} \\
        \mu_{\text{basal-basal}} & \text{if cells } i, j \text{ both basal} \\
        \mu_{\text{apical-basal}} & \text{if cells } i, j \text{ are different types}
    \end{cases}
    \label{eq:differential_adhesion}
\end{equation}

\begin{itemize}
    \item Typically: $\mu_{\text{apical-apical}} = \mu_{\text{basal-basal}} > \mu_{\text{apical-basal}}$
    \item Weaker heterotypic adhesion drives cell sorting and layer stratification
    \item \textbf{Origin:} Differential Adhesion Hypothesis (Steinberg, 1963, 1970)
    \item \textbf{Citation reference:} Steinberg MS (1963), "Reconstruction of tissues by dissociated cells"
    \item \textbf{Bullet points to fill:}
    \begin{itemize}
        \item []
        \item []
        \item []
    \end{itemize}
\end{itemize}

\subsection{Simple ECM Threshold Model}

% Initial simple implementation - ECM as binary yes/no force

\paragraph{ECM Resistance Force}
Initial implementation where ECM exerts force only when cell reaches threshold depth:

\begin{equation}
    \mathbf{F}_{\text{ECM}, i} = \begin{cases}
        -k_{\text{ECM}} (y_i - y_{\text{threshold}}) \mathbf{\hat{y}} & \text{if } y_i > y_{\text{threshold}} \\
        \mathbf{0} & \text{otherwise}
    \end{cases}
    \label{eq:ecm_threshold}
\end{equation}

\begin{itemize}
    \item $k_{\text{ECM}}$ = ECM stiffness parameter
    \item $y_i$ = y-coordinate of cell $i$ (invasion depth)
    \item $y_{\text{threshold}}$ = threshold depth where ECM resistance begins
    \item $\mathbf{\hat{y}}$ = unit vector in y-direction (opposing invasion)
    \item \textbf{Limitations:}
    \begin{itemize}
        \item Binary on/off behavior
        \item No spatial variation in ECM properties
        \item No ECM fiber orientation
        \item Isotropic resistance
    \end{itemize}
    \item \textbf{Bullet points to fill:}
    \begin{itemize}
        \item []
        \item []
        \item []
    \end{itemize}
\end{itemize}

\subsection{Dynamic ECM Agent-Based Grid Model}

% Based on: Painter 2009, Metzcar et al. 2025
% Implements spatially-resolved ECM with fiber orientation, degradation, and remodeling

\paragraph{ECM Grid Representation}
ECM properties stored on spatial grid with spacing $\Delta x$ (typically 25 μm):

\begin{itemize}
    \item Each grid point $(i,j)$ stores:
    \begin{itemize}
        \item $\theta_{ij}$ = fiber orientation angle (radians)
        \item $\rho_{ij}$ = ECM density $\in [0, 1]$
        \item $\alpha_{ij}$ = anisotropy (fiber alignment) $\in [0, 1]$
    \end{itemize}
    \item Bilinear interpolation between grid points for smooth fields
    \item \textbf{Bullet points to fill:}
    \begin{itemize}
        \item []
        \item []
    \end{itemize}
\end{itemize}

\paragraph{Contact Guidance Force}
Cell migration guided by ECM fiber orientation:

\begin{equation}
    \mathbf{F}_{\text{guidance}, i} = v_0 \cdot s \cdot \sqrt{\rho(\mathbf{r}_i)} \cdot \mathbf{\hat{n}}_{\text{migration}}
    \label{eq:contact_guidance}
\end{equation}

where the migration direction is:

\begin{equation}
    \mathbf{\hat{n}}_{\text{migration}} = \frac{\alpha \rho \mathbf{\hat{f}} + (1-\alpha\rho) \cdot 0.3 \mathbf{\hat{f}}_\perp + (1-\alpha\rho) \cdot 0.5 \mathbf{\hat{r}}}{\|\cdot\|}
    \label{eq:migration_direction}
\end{equation}

\begin{itemize}
    \item $v_0$ = base cell speed (μm/min), typically 1.25 μm/min
    \item $s$ = ECM sensitivity parameter (0-1+), default 1.0
    \item $\rho(\mathbf{r}_i)$ = interpolated ECM density at cell position
    \item $\mathbf{\hat{f}}$ = ECM fiber direction unit vector: $(\cos\theta, \sin\theta)$
    \item $\mathbf{\hat{f}}_\perp$ = perpendicular direction: $(-\sin\theta, \cos\theta)$
    \item $\mathbf{\hat{r}}$ = random direction (updated each timestep)
    \item $\alpha$ = anisotropy parameter (1 = fully aligned, 0 = random)
    \item $\rho$ = local ECM density
    \item \textbf{Origin:} Contact guidance concept from Painter 2009; three-factor model from Metzcar et al. 2025
    \item \textbf{Citation references:}
    \begin{itemize}
        \item Painter KJ (2009), "Modelling cell migration strategies in the extracellular matrix"
        \item Metzcar et al. (2025), "Three factors are sufficient to recapitulate heterogeneous cell invasion patterns"
    \end{itemize}
    \item \textbf{Bullet points to fill:}
    \begin{itemize}
        \item []
        \item []
        \item []
        \item []
    \end{itemize}
\end{itemize}

\paragraph{ECM Degradation}
Cells degrade ECM through matrix metalloproteinases (MMPs):

\begin{equation}
    \frac{\partial \rho_{ij}}{\partial t} = -k_{\text{deg}} \sum_{k \in \text{cells near } (i,j)} \delta(\mathbf{r}_k - \mathbf{r}_{ij})
    \label{eq:ecm_degradation}
\end{equation}

\begin{itemize}
    \item $k_{\text{deg}}$ = degradation rate constant (default 0.001 min$^{-1}$)
    \item $\delta(\mathbf{r}_k - \mathbf{r}_{ij})$ = indicator function (1 if cell $k$ at grid point $(i,j)$, 0 otherwise)
    \item Discrete implementation: $\rho_{ij}^{n+1} = \max(0, \rho_{ij}^n - k_{\text{deg}} \Delta t)$ if cell present
    \item Density bounded: $\rho \in [0, 1]$
    \item \textbf{Bullet points to fill:}
    \begin{itemize}
        \item []
        \item []
        \item []
    \end{itemize}
\end{itemize}

\paragraph{ECM Remodeling (Traction-Induced)}
Cells mechanically align ECM fibers through traction forces:

\begin{equation}
    \frac{\partial \theta_{ij}}{\partial t} = k_{\text{remodel}} \cdot \min\left(1, \frac{\|\mathbf{F}_k\|}{F_{\text{ref}}}\right) \cdot \sin(\theta_{\text{cell}} - \theta_{ij})
    \label{eq:ecm_remodeling}
\end{equation}

\begin{itemize}
    \item $k_{\text{remodel}}$ = remodeling rate constant (default 0.05 min$^{-1}$)
    \item $\mathbf{F}_k$ = traction force from cell $k$ (from Equation \ref{eq:contact_guidance})
    \item $F_{\text{ref}}$ = reference force for normalization (typically 10 μm/min)
    \item $\theta_{\text{cell}}$ = cell migration direction angle
    \item $\theta_{ij}$ = current ECM fiber angle at grid point
    \item Stronger forces cause faster ECM realignment
    \item \textbf{Origin:} Traction-mediated ECM remodeling (cell contractility)
    \item \textbf{Citation reference:} Ma et al. (2013), "Fibers in the ECM enable long-range stress transmission between cells"
    \item \textbf{Bullet points to fill:}
    \begin{itemize}
        \item []
        \item []
        \item []
    \end{itemize}
\end{itemize}

\paragraph{ECM Diffusion (Smoothing)}
Spatial diffusion to prevent sharp discontinuities:

\begin{equation}
    \frac{\partial \theta_{ij}}{\partial t}\bigg|_{\text{diff}} = D \nabla^2 \theta_{ij}
    \label{eq:ecm_diffusion}
\end{equation}

\begin{itemize}
    \item $D$ = diffusion coefficient for ECM properties (default 0.01)
    \item Discrete implementation using 4-neighbor stencil
    \item Applied to fiber orientation using circular averaging (angular diffusion)
    \item Also applied to density: $\partial \rho_{ij}/\partial t = D \nabla^2 \rho_{ij}$
    \item \textbf{Bullet points to fill:}
    \begin{itemize}
        \item []
        \item []
    \end{itemize}
\end{itemize}

\paragraph{Bidirectional Fiber Movement}
\begin{itemize}
    \item Cells can move in either direction along fibers: $\pm \mathbf{\hat{f}}$
    \item Direction chosen randomly each timestep with probability 0.5
    \item Reflects biological reality of cells migrating along but not necessarily towards fiber end
    \item \textbf{Bullet points to fill:}
    \begin{itemize}
        \item []
        \item []
    \end{itemize}
\end{itemize}

\subsection{Cell Cycle and Contact Inhibition}

% ContactInhibitionCellCycleModel

\paragraph{Contact-Inhibited Proliferation}
Cell cycle progression depends on local crowding:

\begin{equation}
    \text{Proliferation rate} = \begin{cases}
        r_{\text{max}} & \text{if } V_i > V_{\text{quiescent}} \\
        0 & \text{otherwise}
    \end{cases}
    \label{eq:contact_inhibition}
\end{equation}

\begin{itemize}
    \item $r_{\text{max}}$ = maximum proliferation rate (cell cycle duration)
    \item $V_i$ = current cell volume (or effective area in 2D)
    \item $V_{\text{quiescent}}$ = quiescent volume fraction threshold (typically 0.7 of equilibrium)
    \item Cells enter quiescence (G0) when compressed below threshold
    \item \textbf{Implementation:} ContactInhibitionCellCycleModel in CHASTE
    \item \textbf{Typical cell cycle phases:}
    \begin{itemize}
        \item G1: 8-12 hours (contact-sensitive)
        \item S: 5 hours
        \item G2: 4 hours
        \item M: 1 hour
        \item Total: 18-22 hours
    \end{itemize}
    \item \textbf{Bullet points to fill:}
    \begin{itemize}
        \item []
        \item []
        \item []
    \end{itemize}
\end{itemize}

\subsection{Summary of Key Parameters}

\begin{table}[h]
\centering
\caption{Default parameter values used in simulations}
\begin{tabular}{lll}
\hline
\textbf{Parameter} & \textbf{Symbol} & \textbf{Default Value} \\
\hline
Spring constant & $\mu$ & 15.0 (dimensionless) \\
Damping constant & $\eta$ & 1.0 hour$^{-1}$ \\
Timestep & $\Delta t$ & 0.5 min \\
Cell-cell adhesion & $\mu_{\text{same type}}$ & 0.4 \\
Heterotypic adhesion & $\mu_{\text{diff type}}$ & 0.2 \\
Base migration speed & $v_0$ & 1.25 μm/min \\
ECM sensitivity & $s$ & 1.0 \\
ECM degradation rate & $k_{\text{deg}}$ & 0.001 min$^{-1}$ \\
ECM remodeling rate & $k_{\text{remodel}}$ & 0.05 min$^{-1}$ \\
ECM diffusion coeff. & $D$ & 0.01 \\
Grid spacing & $\Delta x$ & 25 μm \\
Interaction cutoff & $r_{\text{cut}}$ & 50 μm \\
\hline
\end{tabular}
\label{tab:parameters}
\end{table}

\subsection{Notes on Implementation}

\begin{itemize}
    \item All simulations performed using CHASTE (Cancer, Heart and Soft Tissue Environment)
    \item Node-based cell populations with Delaunay triangulation for neighbor detection
    \item Forward Euler time integration for cell positions
    \item Periodic boundary conditions in x-direction for some simulations
    \item Output frequency: every 120 timesteps (60 minutes simulation time)
    \item Typical simulation duration: 2-4 days (2880-5760 minutes)
    \item \textbf{Bullet points to fill:}
    \begin{itemize}
        \item []
        \item []
    \end{itemize}
\end{itemize}
